% 第三章:多模态数据构造范式
\chapter{多模态数据构造范式}

\section{问题定义}
当前多模态大语言模型(Multimodal Large Language Model, MLLM)的训练正面临“\textbf{数据质量而非数据数量}”成为主要瓶颈的阶段性转折。一方面,网络抓取的图文对存在噪声大、对齐弱(Misalignment)、图像质量参差不齐(水印、模糊)等问题;另一方面,人工标注成本高且难以覆盖复杂推理与多样化交互格式。已有研究(如ShareGPT4V、MMEvol、SynthVLM)均表明:\textbf{高信息密度、高对齐度与高推理密度的数据},往往比盲目扩大规模更能有效提升模型能力。

为系统性描述并复用高质量数据生产经验,本文提出\textbf{Auto-Evol}多模态合成数据构造范式:将“合成数据生成”从单步的“图$\rightarrow$文/问答”重构为一个包含\textbf{正反双向链路}、\textbf{任务导向的原子问题生成}、\textbf{代理式生成与进化增强}、\textbf{多维校验与闭环反馈}的工程系统。该范式可统一承载不同任务域的数据生产需求:例如第\ref{chap:gui}章面向GUI的Referring/Grounding能力提升与第\ref{chap:world_knowledge}章面向世界知识的RecQA/KnowQA/FinalQA体系,均可视为Auto-Evol在不同任务路由下的具体实例。

% 注:若后续需要对章节号引用更严谨,可在对应章节加label,这里先用文字描述避免未定义引用。

\section{构造范式框架}
图\ref{fig:auto_evol_pipeline}给出了Auto-Evol范式的闭环框架。该框架从数据源侧同时接入\textbf{原生图片}与\textbf{生成图片}两类输入,通过数据初始化统一格式,在自动/人工协作标注后进行质量过滤;随后将样本路由到“任务导向的原子问题生成”,并按感知、理解、推理三类代理生成多任务指令;最后通过数据增强与数据校验提升多样性、难度与正确性,若校验不过则回流重写/重生成,最终产出可用于监督微调(SFT)与后训练(Post-training)的高质量合成数据。该流程在第\ref{chap:gui}章与第\ref{chap:world_knowledge}章分别以GUI数据构造与世界知识数据构造的形式落地。

\begin{figure}[t]
  \centering
  \includegraphics[width=0.70\linewidth]{figures/chapter3/pipeline.pdf}
  \caption{Auto-Evol多模态合成数据闭环Pipeline:数据初始化--标注--过滤--原子问题生成--增强与校验--回流重写。}
  \label{fig:auto_evol_pipeline}
\end{figure}

\subsection{数据表示}
为了让数据既能被不同生成代理消费,也能被训练框架稳定解析,Auto-Evol采用“\textbf{统一样本骨架 + 任务特定载荷}”的数据表示方式。统一骨架负责记录样本来源、媒介与元信息,任务载荷则承载不同任务的监督信号。

\paragraph{统一样本骨架}
每个样本包含:\textbf{媒体}(图像或多图)、\textbf{来源类型}(原生/生成)、\textbf{元信息}(分辨率、采集时间、领域标签/类别路径)、以及可选的\textbf{结构化标注}(如bbox、OCR文本、实体ID、关系/状态等)。该层保证跨任务的可追溯、可去重与可统计。

\paragraph{任务特定载荷}
任务载荷以“指令--输入--输出”的对话结构组织,支持多轮格式与工具调用字段。根据任务类型可包含:
\begin{itemize}
  \item \textbf{感知载荷}:bbox/OCR/实例级识别等监督信号(对应GUI章中的Grounding与OCR增强,亦对应MMEvol的Fine-grained Perception Evolution);
  \item \textbf{理解载荷}:高密度描述、关系解释、背景知识片段等(对应ShareGPT4V的Better Captions与世界知识章的KnowQA);
  \item \textbf{推理载荷}:显式推理过程(CoT/Think)、可执行程序片段(Code-aided CoT)、以及最终答案(对应世界知识章的Think数据与Synthesize Step-by-Step的“可执行推理”思想)。
\end{itemize}
这种表示方式的核心目标是让模型学习到“从视觉证据到语义到推理”的渐进链路,而非将复杂问题压缩为端到端的黑箱映射。

\subsection{构造流程}
Auto-Evol的构造流程可逐模块映射到图\ref{fig:auto_evol_pipeline}的闭环节点。下文结合第4章(GUI)与第5章(世界知识)的实践,说明各模块的作用与必要性。

\paragraph{(1)数据源:原生图片与生成图片的双向数据流}
Pipeline顶部同时接入两类数据源:
\begin{itemize}
  \item \textbf{原生图片(真实收集)}:来自真实应用截图、公开数据集与搜索收集的高清图片。GUI章使用真实设备分辨率截图以保留小控件细节;世界知识章以“词条$\rightarrow$图片”的方式确保实体可控。
  \item \textbf{生成图片(Text-to-Image)}:当长尾场景稀缺或需要严格对齐时,可采用“文本$\rightarrow$图像”的逆向数据工程。SynthVLM表明,先清洗caption再用扩散模型生成高分辨率图像,并用CLIPScore+SSIM筛选,可从源头提高图文对齐度与图像清晰度,从而以更少数据达到更强效果。
\end{itemize}
双向数据流的价值在于:\textbf{正向流(图$\rightarrow$文)}擅长覆盖真实分布,\textbf{反向流(文$\rightarrow$图)}擅长制造“强对齐、可控分布”,两者互补可显著提升数据的覆盖与可信度。

\paragraph{(2)数据初始化:统一格式与任务路由的前置条件}
“数据初始化”负责将不同来源的样本规范化:补全元信息、统一字段(图片ID/类别路径/实体ID/bbox/OCR等)、去重与基础清洗,并为后续任务分发提供\textbf{Task Router}所需的标签。GUI章的初始化更强调分辨率与布局信息;世界知识章更强调实体锚点与类别层级。该模块对应AgentInstruct所强调的“内容转换流(Content Transformation Flow)”思想:先把原始材料变成适合生成指令的中间表示,才能稳定规模化生产高质量指令。

\paragraph{(3)标注:大模型细粒度标注与人工上下文补全}
Pipeline中部的两条标注路径对应“自动为主、人工兜底”的协作机制:
\begin{itemize}
  \item \textbf{大模型标注生成详细描述}:ShareGPT4V证明,高密度caption能显著增强模态对齐并解锁视觉编码器的微调潜力;在GUI中,该过程体现为自动生成控件描述与候选bbox;在世界知识中体现为包含背景知识的深度描述与候选QA。
  \item \textbf{人工标注收集相关上下文}:对自动标注难以覆盖的条件(页面目的、交互前置、实体歧义消解)进行补全,以减少“幻觉式补写”。该模块在GUI章的人机协同标注与世界知识章的实体锚定中均是质量上限的关键。
\end{itemize}

\paragraph{(4)过滤:相关性与准确性的第一道闸门}
“过滤相关性、准确性”是合成数据可用性的核心保障。MMEvol提出“Instruction Elimination”机制:每轮进化后用LLM比较新旧指令质量,若进化引入幻觉或无意义则淘汰。SynthVLM也通过CLIPScore等指标进行强筛选,强调“质量>数量”。因此,过滤不仅是去噪,更是\textbf{控制数据分布与信息密度}的关键环节。

\paragraph{(5)任务导向的原子问题生成:从能力拆解到可监督信号}
“任务导向的原子问题生成”将复杂任务分解为可组合的原子能力,并生成可监督的子任务集合。该模块正是第4章与第5章数据体系的共同抽象:
\begin{itemize}
  \item GUI章将UI Agent能力拆解为感知(bbox/OCR)、理解(控件语义/关系)、推理(工具与计划)三类原子任务;
  \item 世界知识章将能力拆解为RecQA(视觉识别锚定)、KnowQA(知识注入)、FinalQA(推理贯通)。
\end{itemize}
MMEvol进一步将推理过程抽象为“视觉操作链”(定位、OCR、存在性判断、计算等原子操作),并要求显式生成推理步骤,说明原子化是提高复杂度与降低幻觉的有效路径。

\paragraph{(6)三代理生成:感知/理解/推理的分层合成}
Pipeline中“感知/理解/推理”三块对应分层代理(Perception, Understanding, Reasoning Agents):
\begin{itemize}
  \item \textbf{感知代理}:强化细粒度对齐与长尾信息挖掘(MMEvol的Fine-grained Perception Evolution;GUI的Small Widgets定位;世界知识的实体识别锚定)。
  \item \textbf{理解代理}:生成高密度描述、解释与问答,提高语义覆盖与上下文完整性(ShareGPT4V式caption;GUI的Referring;世界知识的KnowQA)。
  \item \textbf{推理代理}:生成多跳推理与长链路决策数据,并可引入\textbf{代码辅助的思维链(Code-aided CoT)}。Synthesize Step-by-Step表明,将推理拆成“生成步骤+工具执行”,能显著提高数值与逻辑正确性,降低端到端推理的幻觉。
\end{itemize}
该分层结构的作用是把“难任务”拆成“可控生成、可控校验”的组合单元,避免仅靠单一Prompt生成导致任务形式单一与幻觉堆积。

\paragraph{(7)数据增强:细粒度、难度、指令多样性的系统注入}
Pipeline中的“数据增强”可解释为三维增强:
\begin{itemize}
  \item \textbf{细粒度增强}:围绕次要目标、背景细节、空间关系生成更多监督(MMEvol强调长尾物体与细粒度对齐);
  \item \textbf{难度增强}:引入多步推理、约束条件、反事实假设等提升推理密度(MMEvol的Cognitive Reasoning Evolution;AgentInstruct的Refinement Flow);
  \item \textbf{指令多样性增强}:将简单Q\&A改写为代码/JSON/角色扮演等多样格式,提升交互鲁棒性(MMEvol的Interaction Evolution;Self-Instruct的自举式指令扩展也证明多样性对“格式对齐”至关重要)。
\end{itemize}

\paragraph{(8)数据校验:LLM-as-a-Judge + 工具/执行校验的多维验证}
高质量合成数据必须“可验证”。因此Auto-Evol引入多维校验:
\begin{itemize}
  \item \textbf{质量校验}:用LLM-as-a-Judge评估指令清晰度、逻辑性与无歧义性(对应MMEvol的指令淘汰机制);
  \item \textbf{相关性校验}:通过图文匹配分数、实体一致性检查等剔除错配(对应SynthVLM的CLIPScore筛选);
  \item \textbf{执行校验}:对推理类样本执行代码/工具链验证答案一致性(对应Synthesize Step-by-Step的“外部工具执行”确保正确性)。
\end{itemize}

\paragraph{(9)回流重写/重生成:闭环提升与分布自适应}
当某类任务通过率持续偏低时,系统触发右侧“重写”回路:要么对指令/答案进行重写(保持图像不变),要么回流到“生成图片”分支重新合成更契合任务的图像,再进入流程。该闭环使数据系统具备“自我修复”的能力,避免一次性生成导致的质量不可控,也为第4章与第5章的Badcase Mining提供统一接口:把失败样本变成下一轮数据生产的目标分布。

\section{关键技术}
围绕Auto-Evol范式,本文将关键技术归纳为四类,与前两章形成对应关系:
\begin{itemize}
  \item \textbf{高密度对齐(Dense Alignment)}:以ShareGPT4V式高密度caption与GUI/世界知识中的结构化标注为代表,提升视觉--语言对齐粒度;
  \item \textbf{逆向数据工程(Text-to-Image)}:以SynthVLM为代表,先清洗文本再生成图像,用强对齐补足长尾与难例;
  \item \textbf{原子化与显式推理(Atomic + CoT/Tool)}:以MMEvol与Synthesize Step-by-Step为代表,显式生成推理步骤并可借助工具执行,提升复杂任务正确性;
  \item \textbf{进化增强与淘汰机制(Evolution \allowbreak + \allowbreak Elimination)}:结合MMEvol/\allowbreak AgentInstruct/\allowbreak Self-Instruct的迭代式生成与过滤思想,通过“增强--校验--淘汰/回流”持续提高数据质量与多样性。
\end{itemize}
这些技术共同支撑了本文在GUI与世界知识两个任务域中的数据工程落地:前者强调高分辨率细粒度感知与结构化监督,后者强调实体锚定、知识注入与多跳推理链路;二者在Auto-Evol范式下实现统一建模与复用。

\section{本章小结}
本章提出Auto-Evol多模态合成数据构造范式,并以图\ref{fig:auto_evol_pipeline}为中心逐模块阐释其作用:通过原生/生成双向数据流保证分布覆盖与强对齐,通过数据初始化与任务路由统一样本结构,通过自动/人工协作标注提升信息密度,通过过滤与多维校验抑制幻觉并保证正确性,通过原子问题生成与三代理生成系统注入细粒度、语义与推理能力,最终以闭环回流实现可持续迭代。该范式为第4章GUI能力提升与第5章世界知识增强提供统一的数据工程底座,也为后续扩展到图表推理、自动驾驶等场景提供了可复用的方法学框架。

