% 第五章:世界知识
\chapter{世界知识能力提升方案}
\label{chap:world_knowledge}

\section{引言:从视觉描述到世界知识理解}
近年来,大型视觉语言模型(Vision--Language Model, VLM)通过在海量图文对上的预训练,习得了强大的视觉感知与语言生成能力。然而,这类能力往往停留在“\textbf{视觉层面的描述}”,即对颜色、形状、人物姿态等可见信息进行总结,难以进一步回答“图像背后\textbf{隐含的事实、概念与关系}”。

本文将\textbf{世界知识(World Knowledge)}定义为对物理世界、社会文化、历史事件、科学技术等领域的事实、概念、关系与常识的综合掌握,其本质是一个庞大且交织的知识网络。以经典电影海报为例,缺乏世界知识的模型可能只能描述“画面中有一男一女张开手臂”;而具备世界知识的模型应能进一步识别具体人物与作品,并关联到导演、时代背景与象征含义,从而实现从“看见”到“理解”的跃迁。

\section{背景与挑战:世界知识注入的三大困境}
提升VLM的世界知识能力并非简单“增加数据量”即可解决,核心困难可归纳为三类:
\begin{itemize}
  \item \textbf{知识整合与收集(Integration)}:世界知识分散于海量文本语料或结构化图谱中,如何将其转化为VLM可学习、可泛化的监督信号,是数据工程与训练范式的共同挑战。
  \item \textbf{视觉与语言对齐(Alignment)}:文本侧可能已掌握概念与属性(如某类动物的习性),但视觉编码器未必能将\textbf{具体实例(Specific Instance)}与\textbf{抽象描述(Abstract Description)}精确对齐,导致“知道但看不出来”或“看出来但说不对”的错配。
  \item \textbf{多跳推理能力(Reasoning)}:复杂问题往往需要“识别$\rightarrow$检索$\rightarrow$筛选”的多步推理链路,例如“图中导演拍摄的获奥斯卡奖电影是哪部?”要求模型先识别导演实体,再查询其作品集,最后做条件筛选得到唯一答案。
\end{itemize}

\section{方法总览:分类体系与三元组QA数据策略}
针对上述挑战,本文提出一套以数据闭环为核心的世界知识增强方案,主要包括:
\begin{itemize}
  \item \textbf{结构化知识体系}:构建覆盖自然景观、人物、品牌商标等7大领域、40个子类的分层分类框架,保证覆盖面与可扩展性。
  \item \textbf{三元组QA数据策略}:设计RecQA(识别)、KnowQA(纯知识)、FinalQA(推理)三种互补数据格式,用“视觉锚定$\rightarrow$知识注入$\rightarrow$推理贯通”的方式实现深层对齐。
  \item \textbf{基于CoT的性能优化}:通过引入显式推理过程(Think/Chain-of-Thought),缓解多任务训练冲突,显著提升复杂问题的FinalQA指标。
  \item \textbf{迭代式数据构建}:采用多批次、多来源模型参与生成,并通过Badcase Mining定向补齐薄弱实体,形成可持续迭代的闭环。
\end{itemize}

\subsection{知识类别框架(Category Framework)}
为保证世界知识的覆盖面与长尾可扩展性,本文构建分层分类体系,包含7个大类与40个小类,代表性大类如下:
\begin{itemize}
  \item \textbf{自然景观}:河流、山脉、森林、沙漠等;
  \item \textbf{生物}:知名动物、植物、真菌等;
  \item \textbf{建筑景观}:地标建筑、宗教场所等;
  \item \textbf{人物}:政治人物、演艺明星、历史名人等;
  \item \textbf{品牌商标}:商业Logo、产品包装等;
  \item \textbf{文化娱乐}:电影、游戏、动漫、书籍等;
  \item \textbf{科学技术}:特定设备、科学现象等。
\end{itemize}
该框架一方面提供“数据采样的坐标系”(便于覆盖均衡与长尾挖掘),另一方面也作为训练与评测的统计维度,用于分析不同知识域的对齐难度与误差模式。

\section{数据工程:闭环Pipeline与质量控制(对照Pipeline图)}
世界知识增强的关键在于“把分散知识变成可学习的多模态监督”。本文在图\ref{fig:wk_data_pipeline}所示的\textbf{多模态数据合成闭环Pipeline}基础上,构建面向世界知识的自动化数据生产流水线:以“原生图片(真实收集)”与“生成图片(必要时的合成/重写)”为输入端,经由数据初始化、自动/人工标注、过滤、原子问题生成、增强、校验与回流重写,得到可用于多任务训练的最终合成数据。

\begin{figure}[t]
  \centering
  \includegraphics[width=0.68\linewidth]{figures/chapter3/pipeline.pdf}
  \caption{多模态数据合成闭环Pipeline在世界知识数据构造中的映射。}
  \label{fig:wk_data_pipeline}
\end{figure}

\subsection{词条生成与图片收集:从知识域到视觉证据}
对应图\ref{fig:wk_data_pipeline}顶部的“原生图片/生成图片”,本文首先在分类框架下为每个子类生成词条集合,并收集高清图片作为视觉证据:
\begin{itemize}
  \item \textbf{热门与长尾词条}:利用大模型为每个子类生成热门实体与“人类感兴趣的偏门实体”,提高知识覆盖的同时避免仅学习头部概念。
  \item \textbf{多源图像收集}:通过公开数据集与搜索渠道收集高清图片,并保留来源、分辨率、版权与时间等元信息,便于后续质检与去重。
\end{itemize}
在这一阶段,数据工程的目标是“\textbf{实体可控}”与“\textbf{图像高质量}”:实体可控保证知识监督可追溯,图像高质量保证视觉侧对齐不被噪声主导。

\subsection{数据初始化:统一样本结构与知识锚点}
对应Pipeline中的“数据初始化”,本文将原始样本统一为结构化记录,核心字段包括:实体ID(词条)、类别路径(大类/小类)、图像ID与来源、候选别名(同义词/译名)、以及可选的外部知识锚点(如百科条目标题)。数据初始化的目标是为后续自动标注与过滤提供稳定接口,并支持“以实体为中心”的去重与统计。

\subsection{自动标注与QA生成:从描述到三元组任务}
对应Pipeline中的“大模型标注生成详细描述/人工标注收集相关上下文”,本文采用“自动为主、人工兜底”的方式生成两类标注:
\begin{itemize}
  \item \textbf{深度视觉描述(Image Captioning)}:使用先进VLM为每张图片生成包含背景知识的描述,不仅覆盖可见内容,还补充与实体相关的历史、文化或技术背景,从而增强“视觉特征$\leftrightarrow$知识”关联强度。
  \item \textbf{三元组QA生成(RecQA/KnowQA/FinalQA)}:为了解耦感知与推理,本文构造三类互补问答对:
    \begin{itemize}
      \item \textbf{RecQA(Recognition QA)}:依赖视觉识别,问题围绕“这是谁/这是什么/来自哪里”,答案为实体或关键属性,用于将视觉特征锚定到实体;
      \item \textbf{KnowQA(Knowledge QA)}:不依赖视觉输入,仅基于文本知识回答,但与图片实体强相关,用于注入背景知识并形成可检索的知识片段;
      \item \textbf{FinalQA(Reasoning QA)}:需要“先识别再用知识推理”的一跳或多跳问题,用于打通RecQA与KnowQA形成的知识链路。
    \end{itemize}
\end{itemize}

将三类QA映射回Pipeline的“任务导向原子问题生成”,可对应为:RecQA偏\textbf{感知/识别},KnowQA偏\textbf{理解/知识},FinalQA偏\textbf{推理/筛选}。这种拆分避免模型直接学习“图像$\rightarrow$最终答案”的捷径,从而提升泛化与可解释性。

\subsection{过滤、增强与校验:质量优先的闭环迭代}
对应Pipeline中的“过滤相关性/准确性”“数据增强”“数据校验/重写”,本文采用多级质控保证知识可靠与对齐有效:
\begin{itemize}
  \item \textbf{图文相关性过滤}:计算Image--Text相似度或一致性评分,剔除与实体无关的图片、错误匹配的caption与QA;
  \item \textbf{QA质量过滤}:基于逻辑一致性、答案唯一性与可验证性制定评分标准,过滤逻辑不通、答案错误或过于主观的样本;
  \item \textbf{数据增强}:围绕三项能力补强:
    \begin{itemize}
      \item \textbf{细粒度}:为同一实体生成多视角、多属性、多别名表达,提升鲁棒性;
      \item \textbf{难度}:引入对比性问题、干扰选项与多跳链路,提升推理能力;
      \item \textbf{指令多样性}:改写提问方式与对话语气,贴近真实用户提问分布。
    \end{itemize}
  \item \textbf{模型/工具校验与回流重写}:利用强模型复核答案一致性,并对低质量样本进行重写;必要时回流到“生成图片/重写”分支修复描述或问题,形成持续迭代的闭环。
\end{itemize}

\section{训练策略与Badcase迭代:从冲突到CoT对齐}
\subsection{多阶段、多来源数据构建}
训练过程中,本文采用多批次策略逐步扩充数据分布:
\begin{itemize}
  \item \textbf{Batch 1}:使用最强模型构建基础数据,确保高精度与高一致性;
  \item \textbf{Batch 2}:引入不同模型生成以增加知识分布多样性,降低单一模型偏置;
  \item \textbf{Badcase Mining}:针对评估中回答错误的实体与子类定向补充样本,修复薄弱点并提高长尾覆盖。
\end{itemize}

\subsection{初始挫折:多任务数据冲突}
在早期实验中,我们观察到一个反直觉现象:加入深度caption数据对识别指标有显著正向影响,但直接加入FinalQA并未稳定提升推理指标,甚至可能下降。其根源在于:若FinalQA仅提供\textbf{直接答案(Direct Answer)},模型容易记忆“图像特征$\rightarrow$最终答案”的短路映射,跳过中间推导步骤,既不利于泛化,也会与RecQA的识别学习产生干扰。

\subsection{解决方案:引入Think/CoT显式推理数据}
为缓解冲突并提升FinalQA泛化能力,本文将FinalQA升级为“\textbf{先输出推理过程,再输出最终答案}”的格式,即显式CoT(Chain-of-Thought)监督。推理过程通常包含:\textit{识别实体$\rightarrow$检索/回忆背景知识$\rightarrow$按条件筛选$\rightarrow$得到答案}。同时,我们增加KnowQA与FinalQA的“多问一图”密度,以强化知识连接并提高链路覆盖。

\section{实验与分析}
\subsection{基座模型、对比对象与评测基准}
本文以\textbf{Ovis2.5\_9B}作为基座模型,并与开源代表模型进行对比。评测采用\textbf{SimpleVQA}与\textbf{Chinese SimpleVQA}两个基准,分别衡量模型在英文与中文语境下的事实性世界知识掌握程度。

\subsection{评测指标}
本文主要使用两类指标:
\begin{itemize}
  \item \textbf{Recognition Accuracy}:评估模型能否正确识别物体/人物/品牌等实体;
  \item \textbf{Final F1 Score}:评估模型对复杂知识问题的综合推理能力(兼顾正确性与稳定性)。
\end{itemize}

\subsection{结果讨论:Think数据与规模效应}
综合实验可得到三点结论:
\begin{itemize}
  \item \textbf{显著超越基线}:引入结构化世界知识数据后,模型在识别与推理指标上均获得显著提升,体现“视觉锚定+知识注入+推理贯通”的有效性。
  \item \textbf{Think/CoT至关重要}:相较于不带CoT的训练,引入显式推理过程后,Final相关指标出现质变提升,说明“学会如何思考”优于“记住答案是什么”。
  \item \textbf{数据迭代可扩展}:随着多批次数据与Badcase修复的逐步加入,性能呈稳定上升趋势,验证了闭环Pipeline在规模扩展下的可持续性。
\end{itemize}

\section{本章小结与局限}
本章聚焦视觉语言模型缺乏世界知识的痛点,提出了一套基于分类体系与闭环数据工程的增强方法。核心思想是通过RecQA将视觉特征锚定到实体,通过KnowQA注入背景知识,再通过FinalQA打通二者实现多跳推理;同时引入Think/CoT显式推理数据有效缓解多任务冲突并显著提升复杂问题能力。

本方法仍存在两点局限:其一,极长尾知识覆盖仍受制于数据收集与验证成本;其二,参数化知识具有时效性滞后,对于新近事件需要增量数据与再微调。未来可结合检索增强生成(RAG)等机制,提升知识时效性与长尾覆盖的即插即用能力。

